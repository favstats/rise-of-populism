

%<!-- Titleseite -->
\thispagestyle{empty}

%%% redefine \maketitle
\renewcommand{\maketitle}{
	\begin{titlepage}
	
	\Huge
	\begin{center}
		Variants of Populism\\ 
		{\large
		A Cross-National Examination of the Support for Populism in 22 European countries
		}
		\hspace{5cm}
		
		% Author names and affiliations
		\Large
		Marlon Schumacher$^1$, Quynh Nga Nguyen$^2$, Fabio Votta$^3$ \\
		
		\hspace{8pt}
		
		\normalsize  
		$^1$) Graduate Student at the University of Stuttgart\\
		\texttt{M.C.Schumacher@live.de}\\
		$^2$) Graduate Student at the University of Stuttgart\\
        \texttt{qynga.n@gmail.com}\\
		$^2$) Graduate Student at the University of Stuttgart\\
		\texttt{fabio.votta@gmail.com}\\
	\end{center}
	
	\hspace{5pt}
	
	\small
	
This paper seeks to distinguish between and investigate different types of populism. While populism is an intensely debated topic, most scholarship is plagued with conceptual conflations between different variants of populism. To avoid such conceptual confusions, this paper adopts a minimalist definition to identify core features that all subtypes of populism have in common, namely anti-establishment attitudes as well as their opposition to globalization. 

In order to operationalize our conceptual considerations, we use the Chapel Hill Expert Survey dataset and combine it with European Social Survey data to identify respondents that vote for and/or identify with populist parties. Our conceptualization involves two types of populism: progressive and traditionalist populism. For validation we estimate a multinomial logistic regression to test our hypotheses. 

Our models lend support for our theoretical expectations. Economically deprived individuals are more likely to support either variant of populism. Yet individuals who hold traditional values are more likely to support traditionalist populism, whereas the effect goes in the opposite direction for the support of progressive populism.

Further research might be able to build upon our conceptualization and give more attention to the different variants of populism, so as to not conflate the distinct explanatory frameworks that come along with them. 
	
	\scriptsize
	\hspace{5pt}
		\begin{center}
			\hspace{1.5pt}
			In the interest of Open Science, the entire code that was used to generate the content of this paper can be found in the following GitHub Repository: \url{https://github.com/favstats/paper_delib}
		\end{center}
	

	\end{titlepage}
}

%%% automated table of contents
\newcommand{\contents}{
\newpage
\thispagestyle{empty}
\vspace{20mm}
\tableofcontents
}



%%% Title page
\maketitle
\newpage
%\contents
%\clearpage
%\listoffigures
%\clearpage
%\listoftables
%\clearpage	