
%<!-- Titleseite -->
\thispagestyle{empty}

%%% redefine \maketitle
\renewcommand{\maketitle}{
	\begin{titlepage}
		\begin{center}
			\setlength{\parskip}{0pt}
			
			%	    \begin{flushright}
			%	    \colorbox{darkgray}{\color{white}{\Large \textsf{\@headerimg}}}
			%             \end{flushright}
			\begin{multicols}{2}
				\flushleft
                {Prof. Dr. André Bächtiger\par}
				%{Seminar: Transformation of representative democracy\par}
				{Institute for Social Sciences\par}
				{Department of Political Theory and\\ Empirical Research of Democracy\par}
				\begin{flushright}
					\includegraphics[width=7cm]{images/logo_stuttgart.jpg}
				\end{flushright}
			\end{multicols}
			\vspace*{2mm}
			\center
			{\LARGE {Seminar Paper} \par}
			
			\vspace*{10mm}
			
			
			{\fontsize{26}{38} {\bfseries Rise of Populism} \par}
			\vspace*{1mm}
			{ \Large A Cross-National Examination of the Support for Populism in 25 European countries}
			\vspace*{10mm}
			

	\centering
	\begin{tabular}{@{}ccc@{}}
		Author: Fabio Votta, B.A.                    & Author: Marlon Schumacher, B.A.              & Author: Quynh Nga Nguyen, B.A.                                      \\
		Email: fabio.votta@gmail.com                 & Email: M.C.Schumacher@live.de                & Email: qynga.n@gmail.com \\
		Student ID: 2876533                          & Student ID: xxxxxxx                          & Student ID: 2949965                                          
	\end{tabular}

			
			
			\vspace*{5mm}
			
			
			
			
			
			\vspace*{5mm}
			{Date of Submission: 30.03.2018 \par} %\date{xxx}
			
		\end{center}
		\vspace*{2mm}
		\begin{abstract}
			\justifying
			\noindent This seminar paper seeks to investigate deliberation and its relationship to regime support across the world. This is accomplished by exploring the relevant literature and deriving hypotheses from it, which are subsequently tested by using survey data covering 113 countries and 306,047 individual respondents. Given that self-reported regime support is expected to be biased, a weight is applied to account for possible distortions of the data, though results are also reported for the unweighted variable due to the experimental nature of this weight. As this paper is the first known to the authors that examines the effect of deliberation on regime support in a cross-country design, the used deliberation measurement, the Deliberative Component Index from the “Varieties of Democracy”-Project, is examined in a thorough manner and analyses are conducted for its components as well. The analysis finds contradictory evidence for the proposed hypotheses. Deliberation seems to increase regime support first and foremost in democracies, the results in non-democracies and the complete sample are ambiguous and less robust. Furthermore, an exploratory mediation analysis is conducted, to test whether the macro-effect of deliberation on regime support is mediated through democratic performance evaluation on the individual level.  The findings of the analysis suggest that further studies in the field should investigate the relationship between deliberation and regime support as well as democratic performance evaluation in greater detail and find possible methods to remedy bias in self-reported regime support. Moreover, more sensible ways to measure deliberation on the country level are necessary, as it is highly correlated with democracy, although some interesting deviations could be found within the subsamples as well as in regards to the individual components. 
		\end{abstract}
	    \vspace*{2mm}
        \center		
        {\large {Seminar: Transformation of representative democracy} \par}
		
		
		
	\end{titlepage}
}

%%% automated table of contents
\newcommand{\contents}{
	\newpage
	\thispagestyle{empty}
	\vspace{20mm}
	\tableofcontents
}



%%% Title page
\maketitle
\newpage
\contents
\clearpage
\listoffigures
\clearpage
\listoftables
\clearpage

%\clearpage
%
%%<!-- Inhaltsverzeichnisse -->
%\thispagestyle{empty}
%\setstretch{1.15}
%\tableofcontents
%\listoffigures
%
%\clearpage
%\setstretch{1.44}
%<!-- \onehalfspacing -->