\documentclass[]{article}
\usepackage{lmodern}
\usepackage{amssymb,amsmath}
\usepackage{ifxetex,ifluatex}
\usepackage{fixltx2e} % provides \textsubscript
\ifnum 0\ifxetex 1\fi\ifluatex 1\fi=0 % if pdftex
  \usepackage[T1]{fontenc}
  \usepackage[utf8]{inputenc}
\else % if luatex or xelatex
  \ifxetex
    \usepackage{mathspec}
  \else
    \usepackage{fontspec}
  \fi
  \defaultfontfeatures{Ligatures=TeX,Scale=MatchLowercase}
\fi
% use upquote if available, for straight quotes in verbatim environments
\IfFileExists{upquote.sty}{\usepackage{upquote}}{}
% use microtype if available
\IfFileExists{microtype.sty}{%
\usepackage{microtype}
\UseMicrotypeSet[protrusion]{basicmath} % disable protrusion for tt fonts
}{}
\usepackage[margin=1in]{geometry}
\usepackage{hyperref}
\hypersetup{unicode=true,
            pdfborder={0 0 0},
            breaklinks=true}
\urlstyle{same}  % don't use monospace font for urls
\usepackage{graphicx,grffile}
\makeatletter
\def\maxwidth{\ifdim\Gin@nat@width>\linewidth\linewidth\else\Gin@nat@width\fi}
\def\maxheight{\ifdim\Gin@nat@height>\textheight\textheight\else\Gin@nat@height\fi}
\makeatother
% Scale images if necessary, so that they will not overflow the page
% margins by default, and it is still possible to overwrite the defaults
% using explicit options in \includegraphics[width, height, ...]{}
\setkeys{Gin}{width=\maxwidth,height=\maxheight,keepaspectratio}
\IfFileExists{parskip.sty}{%
\usepackage{parskip}
}{% else
\setlength{\parindent}{0pt}
\setlength{\parskip}{6pt plus 2pt minus 1pt}
}
\setlength{\emergencystretch}{3em}  % prevent overfull lines
\providecommand{\tightlist}{%
  \setlength{\itemsep}{0pt}\setlength{\parskip}{0pt}}
\setcounter{secnumdepth}{0}
% Redefines (sub)paragraphs to behave more like sections
\ifx\paragraph\undefined\else
\let\oldparagraph\paragraph
\renewcommand{\paragraph}[1]{\oldparagraph{#1}\mbox{}}
\fi
\ifx\subparagraph\undefined\else
\let\oldsubparagraph\subparagraph
\renewcommand{\subparagraph}[1]{\oldsubparagraph{#1}\mbox{}}
\fi

%%% Use protect on footnotes to avoid problems with footnotes in titles
\let\rmarkdownfootnote\footnote%
\def\footnote{\protect\rmarkdownfootnote}

%%% Change title format to be more compact
\usepackage{titling}

% Create subtitle command for use in maketitle
\newcommand{\subtitle}[1]{
  \posttitle{
    \begin{center}\large#1\end{center}
    }
}

\setlength{\droptitle}{-2em}
  \title{}
  \pretitle{\vspace{\droptitle}}
  \posttitle{}
  \author{}
  \preauthor{}\postauthor{}
  \date{}
  \predate{}\postdate{}


\begin{document}

\subsection{What is Populist about
Populism?}\label{what-is-populist-about-populism}

While populism is an important and intensely debated topic, most
scholarship is plagued with conceptual conflations {[}cf.
@mudde2011voices 1{]}. Despite efforts to avoid such confusions, many
influential scholars continue to use qualifying features of the Right to
describe populism, possibly leading to severe shortcomings in their
empirical analyses. To avoid similar mistakes, this chapter will first
adopt Cas Mudde's clear minimalist definition to identify those core
elements that all subtypes of populism have in common (2.1.1.).
Subsequently, we propose a comprehensive framework to classify European
populist parties along two relevant dimensions: progressive and
traditionalist populism. (2.1.2.)

\subsubsection{A minimalist definition: Moving away from the confusion
between populism and the
Right}\label{a-minimalist-definition-moving-away-from-the-confusion-between-populism-and-the-right}

In almost every handbook about populism, authors would eventually point
out to the concept's contested nature. As Heinisch et al. describe,
``{[}n{]}early as ubiquitous as articles and commentaries on populism is
the assertion that it is a contested concept and difficult to define.
{[}\ldots{}{]} {[}T{]}here have been numerous conceptualisations, which
are themselves derived from several fundamental approaches that differ
{[}\ldots{}{]} in their ideas on whether populism is primarily
ideational, discursive, stylistic, or strategic in nature''
{[}@heinisch2017handbook 22{]}. This contending debate on how to best
define populism is reflected by various empirical studies that emphasize
different and sometimes even contradictory aspects of the phenomenon
{[}cf. @gerring2001social 120{]}. Broadly speaking, there exist three
types of definition for populism. It can be qualified as an
organizational type, as a political communication style or as an
ideology. Especially the latter has gained prominence in scholarly
literature {[}cf. @pauwels2011measuring 99{]}. However, the
differentiation between populism as a communication style and as an
ideology seems artificial at times. Both types distinguish between the
people as opposed to the elite and they both allow for a combination
with other (even diametrically opposed) ideologies. In our understanding
these two types of populism share so many characteristics that the
differentiation between them seems to be a matter of wording rather than
a matter of clear defined features of populism. Most scholars agree on
the ``chameleonic'' character of populism {[}cf.
@taggart2000populism{]}. Some authors, borrowing the notion of a thin
ideology from @freeden1998nationalism, assert that populism can be
combined with other more ``established'' ideologies like liberalism,
nationalism, conservatism, federalism or socialism {[}i.e.
@mudde2016populist 1; @mudde2017populism 19; @albertazzi2007twenty 4{]}.
As Mudde and Kaltwasser emphasize, populism can be ``left-wing or
right-wing, organized in top-down or bottom-up fashion, rely on strong
leaders or be even leaderless'' {[}@mudde2013exclusionary 153{]}. A
serious problem is therefore the confusion between populism and the
Right, as Cas Mudde laments: ``Until now, populism was almost
exclusively linked to the radical right, leading to an incorrect
conflation of populism and xenophobia'' {[}@mudde2016populist 1{]}. This
tendency has to do with the fact that populism gained strength in Europe
with the formation of populist radical right parties in the 1980s {[}cf.
@mudde2013exclusionary 155{]}. Their emergence triggered the blossoming
of a vast scholarly literature -- although focussing almost exclusively
on discussing right-wing populism {[}cf. @de2008pariah{]} while
neglecting the growing impact of their counterpart on the left side of
the political spectrum {[}cf. @lucardie2012populisten{]}.

Despite the already existing thematization of this shortcoming, many
scholars still make the mistake of using right-wing characteristics to
define populism {[}cf. @mudde2007populist{]}. For instance, Inglehart
and Norris, in their analysis on the support for populism in Europe,
justified their definition as follows: ``Cas Mudde has been influential
in the literature, suggesting that populist philosophy is a loose set of
ideas that share three core features: anti-establishment,
authoritarianism, and nativism'' {[}@inglehart2016trump 6{]}.
Considering that the publication they quoted from is called
\emph{``Populist Radical Right Parties in Europe'' (2007)}, Inglehart's
and Norris' statement seems to be remarkably negligent. In this
publication, Cas Mudde unambiguously named authoritarianism and nativism
as ideological features of the populist right and not of populism per se
{[}cf. @mudde2013exclusionary 155{]}. We believe that it is exactly
because of such a theoretical conflation that Inglehart's and Norris'
empirical classification of European parties exhibits serious flaws,
i.e.~by wrongly categorising ostensibly right-wing parties such as the
\emph{German National Democratic Party} or the Hungarian \emph{Jobbik}
as populist left just because they happen to be on the left and right of
the economic policy scale {[}cf. @inglehart2016trump 36{]}. As
Albertazzi points out, ``this insistence on making `populist' and
`extreme right' synonymous or lumping all populists under the `radical
Right populist' banner for ease of comparison {[}\ldots{}{]} is
detrimental to our understanding {[}of{]} {[}\ldots{}{]} populism
itself'' {[}@albertazzi2007twenty 4{]}.

To avoid similar mistakes, we propose a new classification framework,
drawing on a minimalist definition to capture core features that all
subtypes of populism have in common. Following Mudde (who is indeed
``influential in the literature''), we view populism as ``a
thin-centered ideology that considers society to be ultimately separated
into two homogeneous and antagonistic groups, `the pure people' versus
`the corrupt elite', and which argues that politics should be an
expression of the volonté générale (general will) of the people''
{[}@mudde2004populist 543; @mudde2017populism 6{]}. A big advantage of
this minimalist definition is its neutrality, allowing us to analyse
populism independently from the ideological environment in which it
operates {[}@halikiopoulou2012paradox 2{]}. The particularity of Mudde's
definition is the normative distinction made between the ``pure'' people
and the ``corrupt'' elite {[}cf. @mudde2017populism 9{]}. According to
populists, democracy has been perverted by the corrupt elite who act in
their own interests instead of respecting the volonté générale (general
will of the people) {[}cf. @mudde2017populism 16{]}. Populists proclaim
their commitment to fight these corrupt elites as well as other alien
enemies in possession of power in order to give back the sovereignty to
the people {[}cf. @otjes2015populists 60{]}.

Studies of populism were initially focused in a narrow, national or
regional context, first concentrating on the United States and later
expanding to also include Latin America and Europe {[}cf.
@mudde2011voices 1{]}. Recognizing that left-wing populism was widely
neglected in the past, recent scholarship has started to study this
variant of populism {[}i.e. @stavrakakis2014return;
@rendueles2018rise{]}. However, so far this literature tends to conduct
qualitative case studies that concentrate on regional singularities
instead of generalizable tendencies of the phenomenon {[}cf.
@mudde2011voices 1{]}. Left-wing populism is commonly perceived as
geographically limited to Latin America while right-wing populism is
associated with the European political context {[}@hawkins2017populism
267{]}. Given the recently growing importance of left-wing populist
parties in Europe (i.e.~SYRIZA or Podemos), it is important to study the
populist phenomenon taking into account -- but without limiting it to --
the regional context (comprising political, cultural and economic
specificities) that it is embedded in {[}cf. @sorensen2017com 138;
@lanzone2017com 229{]}.

With regard to the populist phenomenon in Europe -- which is this is
paper's focus of study -- two common points shared by all populist
parties regardless of their ideological backgrounds are important to
note. First, as explained above, populists reject establishment parties
that they consider to represent the ``corrupt elite'' acting against the
interests of the ``pure people'' {[}@mudde2017populism 12{]}. Secondly,
populists are consistently opposed to globalization, most notably
represented by the European integration process {[}cf.
@halikiopoulou2012paradox; **Harmsen 2010**; @mudde2007populist;
@hooghe2002does{]}. However, while anti-establishment attitudes and
euroscepticism constitute common denominators of populists from both
sides of the political spectrum, the justification and intensity for
these attitudes vary starkly depending on the ideological orientation of
the populist parties in question {[}**Conti and Memoli 2012 93** in
@van2016united 1184; *? in* @otjes2015populists 60; @mudde2017populism
22{]}. This begs another important question: what, then, distinguishes
the operating logic of different variants of populist parties?

\subsubsection{An essential distinction: Moving towards a comprehensive
conceptualization of progressive and traditionalist
populism}\label{an-essential-distinction-moving-towards-a-comprehensive-conceptualization-of-progressive-and-traditionalist-populism}

The main difference originates from the populists' definition of ``the
pure people'' and ``the corrupt elite'' {[}@mudde2013exclusionary
148{]}. The term populism in itself is derived from the Latin word
populus - literally meaning ``the people''. Populists are however very
ambiguous about `the people' that they intend to represent {[}cf.
@heinisch2017handbook 22{]}. Since `the people' is an ``empty
signifier'' (\emph{Laclau} in @mudde2017populism 9{]}, its signification
varies depending on historical and regional circumstances and its
interpretation differs from party to party {[}cf. @akkerman2017parties
169{]}. As @mudde2017populism assert, ``{[}e{]}ach populist actor
emerges because of a particular set of social grievances, which
influences its choice of host ideology, which in turn affects how the
actor defines `the people' and `the elite'\,'' {[}@mudde2017populism
22{]}. Due to this ambiguity, the dichotomy between the ``pure people''
and the ``corrupt elite'' can be understood from a political, a cultural
or an economic viewpoint {[}cf. @meny2000populismo{]}. Most populists
therefore not only target the political elite, but also other types of
elites like the economic, the cultural or the media elite cf. {[}cf.
@mudde2017populism 12{]}.

As suggested in the previous section, the two most discussed variants
are commonly labelled ``left-wing'' and ``right-wing'' populism. In
recent scholarly literature however, there exist many different notions
to capture these two manifestations of the populist phenomenon.
Alexander and Wenzel, for instance, propose to associate left-wing
populism with ``disaffected liberalism'' and right-wing populism with
``disaffected illiberalism'' {[}@alexander2017myth 8{]}. Kaltwasser and
Mudde, on the other hand, identify a juxtaposition between
``inclusionary populism'' and ``exclusionary populism''
{[}@mudde2011voices 2; @mudde2013exclusionary 158{]}. In this light,
inclusive populism could be seen to focus on inclusion of the
underprivileged (\emph{socioeconomic dimension}), while exclusive
populism primarily focuses on excluding the perceived `others'
(\emph{sociocultural dimension}) {[}cf. @mudde2013exclusionary 167{]}.
In this sense, inclusive populism is considered to exhibit left-wing,
and exclusive populism, right-wing characteristics.
@halikiopoulou2012paradox for their part, consider nationalism to be the
source of populism. Which variant of populism then emerges depends on
whether a civic (left-wing) or an ethnic (right-wing) form of
nationalism is at play {[}cf. @halikiopoulou2012paradox 3{]}. These are
but some of the many existing conceptualizations of populism --
reflecting a certain difficulty to do justice to the diversity of this
phenomenon.

We, for our part, seek to distance ourselves from the left-right labels
and suggest to classify European populist parties along a values
continuum on which \emph{progressive populists} and \emph{traditionalist
populists} are opposing one another. There are two main reasons for as
why we chose to categorize variants of populism along this value
dimension and using these notions. The first reason has to do with the
appropriate naming of public actors. We wish to avoid imposing a label
that the actors in question would not employ to describe themselves.
Populists do not necessarily agree to be pigeonholed as belonging to the
``left'' or to the ``right'' (and especially not to the ``radical-left''
or ``radical-right''). Furthermore, it has been noted that the positions
of some populist parties on political and economic issues are too
incoherent to be consistently categorized as left or right {[}cf.
@huber2017distinct{]}. As for the suggestion to categorize populists as
\emph{``liberal''} or \emph{``illiberal''}, it is important to note that
liberalism is a highly ambiguous concept in itself {[}cf.
@paton2010seeking 9{]}. On the one hand, the ``classic'' or ``Lockean''
version of liberalism insists on individual freedom associated to
private property rights, hence rejecting any form of state intervention.
On the other hand, the ``social'' or ``Millian'' version of liberalism
rejects market inequality in favour of individual agency, hence
encouraging state intervention, economic regulation and resources
redistribution by a welfare state {[}cf. @stephens2017pop 58;
@skorupski1999ethical 215{]} In this respect, both the political right
and the left can be qualified as ``liberal'', depending on the
interpretation of the concept. Furthermore, the term ``liberal'' has an
inextricable political and normative connotation when used within the
setting of liberal democracies. No populist party -- regardless of their
ideological affiliation -- would accept being described as
``illiberal''. Conversely, those populists who feel close to socialist
ideals would shy away from being called ``liberals''. Our second reason
to classify populism along a ``progressivist-traditionalist'' axis is of
conceptual considerations. The notion of \emph{progressivism}, defined
by the Oxford dictionary as ``support for or advocacy of social reform''
{[}@ox2017prog{]}, refers initially to the \emph{Progressive Movement}
that developed in United States in the late 19th century. It emerged as
a reaction to social problems that resulted from the rapid
industrialisation, urbanisation, agrarian depression and financial
recession at the time. Progressivism's main assumption is that
laissez-faire capitalism and excessive individualism has led to social
ills, such as extreme poverty, the formation of slums, increased
prostitution and the exploitation of workers. Progressives hence support
state induced social reforms to help the weak and underprivileged
(especially workers, women and children). The government, responsible
for protecting the ``public interest'' against self-interest,
{[}@prono2008prog 258; @nugent2009progressivism 4{]} is expected to
control finance and industry to make these more accountable to citizens
-- and to fight social injustice in the process {[}cf. @prono2008prog
257{]}. The key value of progressivism is thus ``openness to change''
{[}@nugent2009progressivism 3{]}, and the central belief of progressives
is that ``society could be changed into a better place'' (ibid: 5).

When speaking about progressivism since the late 20th century, it is
notably associated with the ``silent revolution'' that describes a value
change of the post-war generation towards post-materialism {[}cf.
@inglehart1977silent{]}. Post-materialists endorse progressive social
values and new lifestyles, champion self-expression and civil liberties
{[}cf. @inglehart1997modernization 131{]} including cosmopolitanism,
multiculturalism, and a greater tolerance for sexual, ethnic and
religious diversity (cf.~ibid: 23). This value shift brought new issues
on the political agenda such as environmental protection, human rights,
legalisation of abortion, acceptance of homosexuality, and gender
equity, resulting in rising support for left-libertarian parties
(i.e.~the Greens) and progressive movements. {[}cf. @ignazi1992silent;
@inglehart1977silent; @inglehart1997modernization 4f., *23, 240f. in
@inglehart2016trump 3*{]}.

\emph{Progressive populists}, in our understanding, are people who
challenge establishment institutions on the grounds of social justice
and internationalist solidarity. They believe that \emph{``neoliberal''}
transnational elites (represented by institutions like multinational
banks and firms) undermine the people's will by safeguarding a deeply
reactionary society that is in dire need of radical change {[}cf.
@van20176 390{]}. While being essentially tolerant towards cultural,
religious and sexual diversity, progressive populists believe that the
current economic system of free markets is fundamentally defective --
insofar as to benefit the elites to the detriment of the
underprivileged, ordinary people {[}cf. @march2005s 25;
@bornschier2010new{]}. Examples that progressive populists frequently
use to criticize the dysfunctionality of the status quo are the
financial crisis of 2008 or the European debt crisis. ``Neoliberal''
policies pursued by financial elites are perceived as widening the gap
between rich and poor and as a danger to welfare state commitments
{[}cf. @van2016united 1184{]} \emph{Progressive populists} reject
globalisation for fear of capital flight and depreciation of labour
{[}@frieden1996impact; @hawkins2017populism 389{]}. They therefore
oppose the EU -- an institution considered as a means of action for
these global neoliberal elites (i.e. ``Goldman Sachs' revolving doors in
Brussels'') {[}cf. @hooghe2004does 128; @march2015out{]}.

The antithesis to \emph{progressivism} is \emph{traditionalism}.
Tradition, in its broadest sense, is ``anything which is typical of the
past, customary, or part of a cultural identity'' {[}@allison2009trad
537{]}. Traditionalism is the ``propensity to revive or defend
traditions against non‐traditional beliefs and values'' (ibid), and
traditionalists are people who highly value traditions. In its initial
meaning, traditionalism refers to the idea that established institutions
and order, most notably represented by the church and the monarchy,
should be protected against radical revolutions. {[}cf. @allison2009trad
537{]} In a wider sense, traditionalism's object of protection can
include diverse items ranging from religious orientation, sporting
customs, linguistic practices to dietary habits''
{[}@allison2009trad{]}. The key value of traditionalism is the
preservation of stability, and the central belief of traditionalists is
that society must defend the customs they hold dear against chaotic
change {[}cf. @allison2009trad 537{]}. Examples for progressive
doctrines that oppose political traditionalism are Marxism or liberal
capitalism, which -- in their rationalism -- are inherently intolerant
towards traditional values and practices. These doctrines were however
unable to provide people's lives with cultural or spiritual meaning,
which is important to give them a sense of security, stability and
achievement. This contributes to the appeal of traditionalism, often
manifesting through divers forms of nationalist or religious renaissance
(cf.~ibid). Traditionalism of the late 20th century refers to
counterrevolutionary reactions to the ``silent revolution''. Supporters
consist primarily of white men, elders and those left behind by
globalisation, who fear that the rise of progressive demands will
endanger their material status, benefits and privileges. They also see
the strengthening of post-materialist values as a threat to once
prevailing cultural norms that they cherish. Traditionalists value
tradition and stability over change, believing that the government
should apply strict measures to protect their social and cultural
identity against morally corrupt influences. For them, security and
order has priority over universal liberal norms {[}cf.
@inglehart2016trump 7, 13{]}.

\emph{Traditionalist populists}, in our understanding, are people who
challenge establishment institutions to protect their sociocultural
identity. They draw their support from people who believe that
cosmopolitan liberal elites undermine national unity and that societal
change has gone too far. They usually endorse nationalist and
authoritarian values {[}@mudde2010ideology, @mudde2013exclusionary
155{]}, emphasizing the necessity to have strong leaders repelling
``external threats'' to the nation state. This tendency manifests itself
most notably in ``ethno-centric'' discourses
{[}@hainsworth2008extreme{]} and calls for harsh policies concerning
asylum and immigration issues {[}@mudde2007populist; @mair1998party{]}.
\emph{Traditionalist populists} reject globalisation, which they see as
a danger for national and local identities {[}cf. @van20176 390{]}. This
logic translate into their euroscepticism in defence of national
sovereignty and cultural homogeneity against alien influences {[}cf.
@van2016united 1184{]}.

Summing up, while all populists are at odds with political elites,
progressive populists especially target the economic elite. By contrast,
traditionalist populists are characterized by nativism and focus on
arguments around cultural antagonism {[}@akkerman2017parties 170; cf.
@mudde2007populist 18 ff.{]}. Simply put, progressivists frequently
enter in a marriage of convenience with some type of `socialism' while
traditionalists do the same with some type of `nativism'
{[}@mudde2016populist 1; @mudde2017populism 21{]}. Having established a
hierarchical framework to categorize European populist parties along a
populist-establishment and a progressive-traditionalist dimension, we
now proceed to explain the growing support for European populist
parties.

\subsection{What explain's Populist's
Populism}\label{what-explains-populists-populism}

Literature on the electoral success of populist parties is dominated by
two broad strand of causal theorizing: (1) the Durkheimian sociological,
``mass society'' thesis that focusses on feelings of cultural identity
loss and (2) the Downsian rational-choice, ``economic'' approach that
builds upon materialist conceptions of representative politics. Despite
significant differences within both strands with regard to the
independent variables that they use (i.e.~modernization, globalization
or electoral rules), the causal mechanisms behind their arguments are
similar {[}@hawkins2017populism 268 f.{]}. Within the latter group of
Downsian rationalist accounts, Hawkins et al. narrow down
interpretations of populist electoral success down to:

\begin{enumerate}
\def\labelenumi{(\alph{enumi})}
\item
  a medium-term inability of establishment parties to immediately
  address socioeconomic change as demanded by the electorate, or
\item
  a long-term response of citizens to a fundamentally corrupt governance
  system (both are ``demand-side'' accounts), or
\item
  the strategic exploitation of electoral rules by parties resorting to
  populist techniques (a ``supply-side'' account) {[}cf.
  @hawkins2017populism 270 f.{]}.
\end{enumerate}

As @hawkins2017populism notice, scholarship so far has ``given little
attention to the causes of populism at the individual level'', with
``research say{[}ing{]} little about the mentality of populist voters or
the cognitive processes that lead people to join populist forces''
{[}@hawkins2017populism 267{]}. To explain individuals' support for
certain populist parties, this study will focus on the ``medium-term
structural change'' account, which interprets populism as the result of
citizens' normative attitudes and material situation when faced with
socioeconomic change {[}@hawkins2017populism 270 f.{]}.

Populism is not a new, but a rather recurring phenomenon. Support for
populist parties grow especially during periods of major structural
transitions because formerly established institutions and order are
thrown into a maelstrom of uncertainty, a sort of crisis situation
{[}cf. @kelly2017pop 511{]}. As the primary catalyst for socioeconomic
change, industrial modernization serve as the explanatory factor for the
rise of anti-establishment forces during the late 19th and early 20th
century. Industrialisation and urbanization restructured the division of
labour, and in the process they also alter the relationship between the
state and the individual. Society becomes more atomized, moving away
from familiar vehicles of social integration, like the church or the
family, towards more bureaucratic and anonymous institutions.
Individuals feeling socially disintegrated in this modernization process
turn to populist parties who promise to bring back the ``good old days''
{[}cf. @hawkins2017populism 269{]}. In the post-industrial era of the
late 20th and early 21st century, globalization as the new trigger for
structural change accounts for the strong uptrend of contemporary
populism. This argument, now commonly known as the \emph{globalization
losers thesis}, was first proposed by betz1994radical, who reasons that
globalization reorganized the economy and society to the benefit of
some, but also to the detriment of others {[}cf. @betz1994radical{]}.
Globalization essentially depicts a disconnection between social and
spatial relations {[}cf. @scholte2005globalization{]}. In a figurative
manner, it is described as ``the process of world shrinkage, of
distances getting shorter, things moving closer {[}\ldots{}{]},
pertain{[}ing{]} to the increasing ease with which somebody on one side
of the world can interact, to mutual benefit, with somebody on the other
side of the world'' {[}cf. @larsson2001race 9{]}. From an economic
perspective, it is understood as a process of ``markets and people
around the world {[}\ldots{}{]} becoming more integrated over time''
{[}@sen2001if n.p.{]}. In this sense, factors of production are becoming
increasingly mobile, with especially four aspects concerned: goods
(trade), money (capital), people (labour), and ideas (knowledge) {[}cf.
@IMF2000global{]}. As globalization brings about new cultural, social
and economic inputs, it triggers fear and hostility for several,
distinct reasons. While the accelerating movement of goods and money are
worrisome to individuals who depend heavily on socioeconomic stability,
the increased mobility of people and ideas worry primarily those who
want to safeguard a homogenous, culturally inclusive society. Both the
socioeconomic and the sociocultural aspect of globalization play
therefore distinctive roles in the strengthening of support for populist
parties. While economic grievances contributes to the general electoral
success of populist parties (2.2.1.), differing values and attitudes
internalized by individuals will determine which type of populist party
they are likely to support (2.2.2.).

\subsubsection{The Socio-Economic Aspect of
Globalization}\label{the-socio-economic-aspect-of-globalization}

Today's post-industrial age, characterized by a drop in industrial
activities and by a growing demand of services, gives new opportunities
to the ``winners'' of globalization who stand out as being ``flexible,
professional, and entrepreneurial''. Those who do not meet these
criteria -- often ``the unemployed, the underemployed, the unskilled,
and those whose jobs are threatened by advancing technology' -- are the
`losers' of globalization'' {[}@betz1994radical{]}. Globalization in the
economic and financial sphere gives some producers, investors and
workers better conditions to make profit while making it more precarious
for others in terms of incomes, jobs security and welfare benefits
\emph{(Frieden and Rogowski 1996 in Verrbeek and Zaslove 2017: 390)}.
Because these ``losers'' feel abandoned by mainstream parties, which
have consistently supported pro-market policies associated with
globalization, they turn to populist parties for democratic
representation {[}cf. @betz1994radical{]}. As measured by the KOP
Globalization Index, the movement of goods, people, capital and
knowledge has steadily increased since the 1970s \emph{(cp.~KOF Swiss
Economic Institute)}. As a result, patterns of economic competition have
changed, which in turn creates an increasing number of people unable to
offer the sought-after skills and know-how for rewarding jobs. Well-paid
work for hardworking, but less qualified workers gradually disappeared,
especially in developed countries. This tendency frustrates especially
those having difficulties to be mobile (i.e.~older people or low-skilled
people facing language barriers). Their fear of losing their jobs if
exposed to fierce competition in the global labour market is therefore
particularly great {[}cf. @spruyt2016supports 337{]}. As summed up by
Rodrik:

\begin{quote}
``Globalization drove multiple, partially overlapping wedges in society:
between capital and labor, skilled and unskilled workers, employers and
employees, globally mobile professionals and local producers,
industries/regions with comparative advantage and those without, cities
and the countryside, cosmopolitans versus communitarians, elites and
ordinary people. It left many countries ravaged by financial crises and
their aftermath of austerity'' {[}@rodrik2018populism 12{]}.
\end{quote}

Especially events like the financial crisis of 2007-2008 serve for some
as proof for the degeneration of excessive globalization. According to
the OECD, such crises are more likely to occur when the degree of
uncontrolled capital flows is high {[}cf. @oecd2011risk 49{]}. The EU,
an institution embodying cultural and economic integration {[}cf.
@eu2018global{]}, is seen as a vehicle for globalization and hence as a
danger to the status and wellbeing by many people. Above all, EU
institutions are especially disliked because they deny member states the
freedom to adopt their own economic policies {[}@hawkins2017populism
271{]} Despite awareness for these public concerns, ``the economic
anxiety, discontent, loss of legitimacy, fairness concerns that are
generated as a byproduct of globalization rarely come with obvious
solutions or policy perspectives'' {[}cf. @rodrik2018populism 12{]}.
Whether the hypothesis that globalization causes economic misery is
objectively true does not matter in this case. As establishment parties
fail to fix the problem, people who perceive themselves as
``globalization losers'' see in anti-establishment and
anti-globalization parties the only solution to address their fears.

On the other hand, it has been questioned by some authors whether a
higher degree of globalization really leads to major social problems. As
Agéndor and Florian argue, globalization can induce higher net income
inequality, higher rates of unemployment and poverty, but mainly in the
beginning of the process. In the long run, a higher level of
globalization results in the decline of poverty and income inequality
{[}cf. @agenor2004does; @dorn2018globalization{]}. However, ``{[}i{]}n
the long run we are all dead'', as John Maynard Keynes cleverly points
out, ``{[}e{]}conomists set themselves too easy, too useless a task if
in tempestuous seasons they can only tell us that when the storm is long
past the ocean is flat again'' {[}@keynes1923some 80{]}. Since
especially those in an economically precarious situation depend on their
jobs for daily survival, they are not in a position to wait for the
long-term positive impact of globalization to manifest. It is therefore
plausible to assume the following:

\begin{quote}
H1: The more precarious the economic situation of an individual, the
higher the probability of that individual supporting populist parties.
\end{quote}

While globalization is an ``economic or technical phenomenon'', it is
also ``a cultural evolution'' at the same time. This is why its
sociocultural aspect also influence the perception that citizens have of
populist parties.

\subsubsection{The Socio-Cultural Aspect of
Globalization}\label{the-socio-cultural-aspect-of-globalization}

Individuals dissatisfied with current establishment elites and resenting
the ills of globalization are more likely to seek anti-status quo
alternatives by supporting populist parties. However, rational citizens
do not remain indifferent about whom they entrust the task of dethroning
the `corrupt elite' from the party political power stronghold. Given a
choice, these individuals will obviously slant towards the populist
party that they feel ideologically closer to. Whether citizens support
the progressive or traditionalist variant of populism depends on
sociocultural values, norms and practices that they have internalized.

Culture, in the first place, is constituted by collective values and
norms and therefore plays an important role for social coherence. It is
part of a ``collective consciousness'', responsible for ``form{[}ing{]}
a moral glue that results in social integration.'' The feeling of
disintegration and normlessness, also called anomie by Durkheim, occurs
especially during periods of transitions like the 19th century's
industrial modernization and today's resurgence of globalization
{[}@hawkins2017populism 269{]}. Culture, according to Kroeber and
Kluckhohn,

\begin{quote}
``consists of patterns, explicit and implicit, of and for behavior
acquired and transmitted by symbols, constituting the distinctive
achievements of human groups, including their embodiment in artifacts;
the essential core of culture consists of traditional (I.e.,
historically derived and selected) ideas and especially their attached
values; culture systems may, on the one hand, be considered as products
of action, on the other, as conditioning elements of future action''
{[}@kroeber1952culture 181{]}.
\end{quote}

Cultural norms and practices are important as they provide people's
lives with a sense of security, stability and achievement {[}cf.
@allison2009trad 537{]}. Traditionally understood, it is bound to a
specific territorial boundary. However, an increased level of
globalization also means that cultural practices and experiences become
gradually deterritorialized {[}cf. @tomlinson2012cultural 2 ff.{]}. As a
consequent, certain aspects of the local culture may be ``diluted'' or
even disappear while particularities become stronger. This constitute a
worrying development especially for individuals who cherish traditional
ways {[}cf. @nijman1999cultural 150{]}.

Furthermore, cultural globalization goes hand in hand with an active
promotion of a cosmopolitan identity {[}cf. @spruyt2016supports 337{]}.
Cosmopolitanism is part of a progressive mentality, which is rejected by
those who fear a loss of values and traditions. Images of an invasion by
foreigners, who pose a threat to cultural norms and therefore to social
cohesion, often serve as a strong stimulus for anti-immigration
sentiments {[}cf. @mudde1999single 188 ff.{]}. This is why identity has
gained an important place in the political agenda, especially of those
who endorse traditional-conservative and materialist values
{[}@hawkins2017populism 390{]}. It is therefore plausible to assume the
following:

\begin{quote}
H2: The more culturally exclusive the values of an individual, the
higher the probability of that individual supporting traditionalist
populist parties.
\end{quote}

Opposing these culturally exclusive individuals are those whose cultural
norms and values are fundamentally cosmopolitan. These individuals
mainly belong to the post-war generation, which grew up endorsing
post-material, universal ideal about human rights, rule of law,
tolerance and individual freedom from traditional constraints
{[}@jansen2011populist 297; @hawkins2017populism 390{]}. As Calhoun
explains, ``cosmopolitanism has become an enormously popular rhetorical
vehicle for claiming at once to be already global and to have the
highest ethical aspirations for what globalization can offer''
{[}@calhoun2008cosmopolitanism 209{]}. A cosmopolite, which derives from
the word kosmos (world, universe) and polis (citizen), describes a
person who is culturally inclusive. Ingram, citing emancipatory
struggles such as the 1848 ``Springtime of Nations'' or the Communist
Manifesto's call to international solidarity amongst workers, argues
that populism can have a universalistic, cosmopolitan and inclusive
character. Populism is thus not always territorially or ethnically
exclusive. This is why he criticizes the conception that populism is
always xenophobic and cosmopolitanism is always elitist (\emph{Ingram
2013} in @hawkins2017populism 654 f.{]}. It is therefore plausible to
assume the following:

\begin{quote}
H3: The more culturally inclusive the values of an individual, the
higher the probability of that individual supporting progressive
populist parties.
\end{quote}

It is however important to note that the socioeconomic and sociocultural
dimensions of globalization are not completely independent from one
another. Economic and sociocultural factors may interact in a way that
intensify support for populist parties. For instance, the new labour
structure induced by contemporary globalization has produced ``a
splintered and atomized workforce; without powerful unions to reinforce
a new sense of class identity, individuals find themselves powerless to
mobilize'' {[}@hawkins2017populism 269{]}. As a result, feelings of
identity loss, alienation and helplessness develop amongst those who are
unable to cope with these changes {[}cf. @hawkins2017popcau 269{]}. In
this case, economic globalization has also contributed to the widening
of sociocultural cleavages. It is also well-established that especially
individuals who are in a precarious economic situation resent
immigrants, whom they might lose their jobs to, or whom they see as a
cost to welfare benefits. In this case, economic insecurity reinforces
anti-immigrant feelings {[}@decleen2017pop 349{]}. It is therefore
plausible to assume the following:

\begin{quote}
H4: Economic insecurity and cultural values interact in a way that
increases the probability of an individual supporting populist parties
\end{quote}


\end{document}
